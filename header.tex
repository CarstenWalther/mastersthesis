%!TEX root = thesis.tex
%\documentclass[10pt,a4paper,onecolumn,twoside,openright]{book}
\let\accentvec\vec
\documentclass[10pt,a4paper,onecolumn,twoside,openright,titlepage]{svmono}
\let\spvec\vec
\let\vec\accentvec
% added by me
\usepackage{siunitx}

\usepackage{amssymb}
\usepackage[english]{varioref}		% Vario REF \vref
\usepackage{hyperref}			% Clickable links
\usepackage{url}
%\usepackage{graphics}
\usepackage{graphicx}							% Graphic support for eps
\usepackage[utf8]{inputenc}		% Input Encoding auf Windows einstellen
\usepackage[T1]{fontenc}					% Umlaute unterstützen
%\usepackage{ams}
\usepackage{textcomp}							% TC fonts
\usepackage{color}
%\usepackage{alltt}								% includes verbatim from text files
\usepackage{lmodern}							% Schriftanpassungen
\usepackage{multicol}							% Mehrere Spalten im Text verwenden
%\usepackage[bottom]{footmisc}			% Fußzeilen
\usepackage{pifont} 							% for fancy bullets
\usepackage{setspace}							% Zeilenabstand
\usepackage{fancyhdr}							% Kopfzeilen
\usepackage{makeidx}							% Index
\usepackage[english]{babel}				%	nationale Datumsformate, etc.
\usepackage{nomencl}							% Erstellt Abkürzungsverzeichnisse
%\usepackage[style=long,border=none,header=plain,cols=2,number=none,hypertoc=true,acronym=false,global=true]{glossary}   %
\usepackage{tabularx}							% Tabellen mit Type X haben automatische minipage mit Zeilenumbruch
\usepackage{multirow}
\usepackage{rotating}             % Rotieren von float elementen
\usepackage{booktabs}							% Tabellen verschönern mit toprule/bottomrule
\usepackage{colortbl}							% farbige Tabellen
\usepackage[numbers, sort]{natbib}								% Naturwissenschaftliches Zitieren
%\usepackage[small, normal, bf, up]{caption2} % Paket für schönere Bildunterschriften
%\usepackage[nottoc]{tocbibind}		% Literaturverzeichnis in Inhaltsverzeichnis aufnehmen
\usepackage{listings} % Programm-Listing
\lstset{language=C}
\usepackage{float}
\usepackage{amsmath}
\usepackage{parskip}
%\usepackage[toc,nonumberlist]{glossaries}
\usepackage[acronym,nomain]{glossaries}
\usepackage[toc]{appendix}
\usepackage{algorithm}
\usepackage{chngcntr}
\usepackage{algpseudocode}
%\usepackage{algorithmic}
\usepackage{xspace} % solves problem with missing space after \newcommand
% used for \cref command
\usepackage[capitalise,noabbrev]{cleveref}
% dont give equations a name
\crefname{equation}{}{}

% http://homepage.ruhr-uni-bochum.de/Georg.Verweyen/pakete.html
\usepackage{ellipsis}
\usepackage{microtype}

% used for graphs and diagrams
\usepackage{tikz}
\usetikzlibrary{shapes,arrows}
% used for plots
\usepackage{pgfplots}
\pgfplotsset{compat=1.7}

%used for newline in table cell
\usepackage{makecell}

%\counterwithout{footnote}{chapter}

\fancypagestyle{plain}{										% Redefining the plain style
	\fancyhf{}
	\renewcommand{\headrulewidth}{0pt}
	\renewcommand{\footrulewidth}{0pt}
}

%Kopf- und Fußzeile
\pagestyle{fancy}																		% {fancyhdr}: besetzt Kopf- und Fusszeilen mit definiertem Inhalt
\fancyhf{}																					% clear all header and footer fields
																										% im Stil slanted-shape = geneigt; nouppercase = klein geschrieben
\fancyhead[LE,RO]{\bfseries \thepage}								% LE=left-even + RO=right-odd: \thepage=Seitenzahl
\fancyhead[LO]{\bfseries \nouppercase{\rightmark}}		% LO=left-odd: \rightmark=lower-level sectioning information
\fancyhead[RE]{\bfseries \nouppercase{\leftmark}}		% RE=right-even: \leftmark=higher-level sectioning information
\renewcommand{\headrulewidth}{0.5pt}								% Linie zum Abtrennen der Kopfzeile mit entsprechender Breite
\renewcommand{\footrulewidth}{0pt}									% Linie zum Abtrennen der Fusszeile mit entsprechender Breite (0=keine Linie)
%\setlength\headheight{13.6pt}
\headheight 13.6pt																	% wegen Schriftgröße 11pt muss die Höhe der Kopfzeile vergrößert werden

%\setacronymnamefmt{\gloshort}  % setzt den ersten Wert im Abkürzungsverzeichnis auf die Abkürzung selbst
%\setacronymdescfmt{\glolong: \glodesc}
%\makeacronym
%\newglossarystyle{modsuper}{
%  \glossarystyle{super}
%  \renewcommand{\glsgroupskip}{}
%}
\makeindex
%\makeglossary
%\makeglossaries

% Punkte zw. Abkürzung und Erklärung für Abkürzungsverzeichnis
\setlength{\nomlabelwidth}{.20\hsize}
\renewcommand{\nomlabel}[1]{#1 \dotfill}
%\makenomenclature


%\bibliographystyle{unsrtnat}
%\bibliographystyle{unsrt}
\bibliographystyle{acm}
%\bibliographystyle{ieeetr}
\newcommand*{\Mail}[1]{\href{mailto:#1}{\protect\url{#1}}}
\newcommand*{\MailTitle}[2]{\href{mailto:#1@#2}{\protect\url{#1}}}
\newcommand{\keywords}[1]{\par\addvspace\baselineskip\noindent\keywordname\enspace\ignorespaces#1}
\newcommand{\biburl}[2]{\url{#1} (last visited #2)}
\newcommand{\itembf}[1]{\item \textbf{#1}}
\newcommand{\blankpage}{\clearpage{\pagestyle{empty}\cleardoublepage}}

%\def\thechapter{\Alph{chapter}}
%\def\thesection{\arabic{section}}

\onehalfspacing % {setspace}: setzt den Zeilenabstand auf 1,5
% Ränder definieren
\oddsidemargin 1in
\evensidemargin 0.5in

\renewcommand{\arraystretch}{1.5}
\renewcommand{\tabcolsep}{6pt}
\selectlanguage{english}

\definecolor{grey}{rgb}{0.85,0.85,0.85}
\definecolor{hpi_orange}{rgb}{0.88, 0.43, 0.20}
\definecolor{hpi}{rgb}{0.537,0.063,0.165}

\lstset{breaklines=true,
	numbersep=5pt,
	stepnumber=1,
	numbers=left,
	captionpos=b,
	float=htbp,
	basicstyle=\ttfamily\fontsize{9}{12}\selectfont
}
\lstset{frame=tb, aboveskip=0.5cm, belowskip=0.5cm}

\newcommand{\todo}[1]{
	\textbf{\textcolor{hpi}{TODO #1}}
}