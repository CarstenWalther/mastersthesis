
\chapter{\pricewars Platform}
%todo picture from UI

The \pricewars platform is an open framework for simulating online marketplaces \cite{edoc2017pricewars}.
Users can test their pricing strategies under competition in an sandbox environment.
These tests would be time-consuming and possibly costly on a real online marketplace.
A strategy's performance can be compared to other strategies using the evaluation tools provided in a web UI.

The \pricewars platform is a microservice architecture and consists of the following components:
%todo show architecture picture
\todo{bold description bullet point}
\begin{description}
	\item [Marketplace]
		The marketplace is a central place for merchants to offer products and for consumers to buy them.
		The marketplace makes open offers accessible to both merchants and consumers.
		It handles registration of merchants and consumers.
		Each action on the marketplace is written to a logging service.
		The marketplace notifies a merchant, whenever it sells a product.
	\item [Merchant]
		Merchants want to sell products by offering them on the marketplace.
		They can set any offer price.
		The price choice will influence their chances to sell products.
		A merchant's pricing strategy can be a simple rule-based strategy like ''always undercut the cheapest competitor'' or a complex data-driven strategy that analyses the consumer behavior or competitors' strategies.
		%not this or that strat, everything in between
		The \pricewars platform supports data-driven merchants by providing historical market and sales data.
	\item [Consumer]
		The consumer service creates a stream of consumers who visit the marketplace, inspect available offers, and buys a product.
		In case the consumer does not find any acceptable offers, he leaves the marketplace without buying a product.
		The consumer service implements different buying behaviors, which can be configured to be enabled, disabled, or mixed together.
	\item [Producer]
		A producer provides merchants with products.
		The products can be of varying quality.
		Merchants pay a certain amount of money per product.
		This amount is specified by the producer.
	\item [Management UI]
		The management UI is a web interface that allows to control and observe the simulation.
		Marketplace, merchants, consumer, and producer can be configured with the management UI.
		With this level of control, it is possible to test how merchants react on changing market situations. E.g. how fast adapts a merchant its behavior if the number of consumers doubles.
		Various charts show merchant and consumer actions as well as merchants' short- and long-term performance.
	\item [Kafka Reverse Proxy]
		This services provides merchants with data about past market situations and sales.
		The data is filtered --- a merchant gets only information about his own sales --- and transformed into the CSV format.
		Additionally, the management UI gets continuous updates for its charts from the kafka reverse proxy.
		
\end{description}

Additionally, the services Kafka, Zookeeper, Flink, Postgres, and Redis are used to store and process data.
%This sentence is to general. Maybe explain each service in detail.
The platform's source code and documentation is available on \url{https://github.com/hpi-epic/pricewars}.
