
\chapter{\pricewars Platform}
\todo{ picture from UI}

The \pricewars platform is an open-source framework for simulating online marketplaces \cite{edoc2017pricewars}.
Users can test their merchant pricing strategies under competition in a sandbox environment.
These tests would be time-consuming and possibly costly on a real online marketplace.
A web UI provides ways to configure and control the simulation.
Additionally, it contains tools to evaluate a merchant's strategy.
The platform is a microservice architecture and consists of the following components:

\todo{show architecture picture}

\todo{bold description bullet points}
\begin{description}
	\item [Marketplace]
		The marketplace is this platform's central service.
		Merchants use it to offer their products and consumers use it to buy products.
		The marketplace manages merchant and consumer accounts.
		They have to register first, before they can perform actions on the platform.
		Both, merchants and consumers can see all open offers on the marketplace.
		Merchants can adapt their prices according to competitors' prices with this information.
		Consumers inspect these offers to find their preferred offer.
		Each event that happens on the marketplace is written to a logging service.
		This data is processed to e.g. calculate a merchant's revenue.
		The marketplace notifies a merchant, whenever it sells a product.
	\item [Merchant]
		Merchants want to sell products by offering them on the marketplace.
		They can set any offer price.
		The price choice will influence their chances to sell products.
		For example, a higher price usually results in less sales.
		A merchant's pricing strategy can be a simple rule-based strategy like ''always undercut the cheapest competitor'' or a complex data-driven strategy that analyses the consumer behavior or competitors' strategies.
		Of course, also complex rule-based strategies or a hybrid of both approaches are possible.
		The \pricewars platform supports data-driven merchants by providing historical market and sales data.
		A merchant can be written in any programming language as long as it complies with the platform's RESTful API.
		The available merchant implementation\footnote{\url{https://github.com/hpi-epic/pricewars-merchant}} can be used to quickly build a merchant with a custom strategy.
	\item [Consumer]
		The consumer service creates a stream of consumers who visit the marketplace, inspect available offers, and buys a product.
		In case the consumer does not find any acceptable offers, he leaves the marketplace without buying a product.
		The consumer service implements different buying behaviors, which can be enabled, disabled, or mixed together.
	\item [Producer]
		A merchant can restock his inventory with new products from the producer.
		Products can be of varying quality.
		Merchants pay a certain amount of money per product.
		This amount is specified by the producer.
	\item [Management UI]
		The management UI is a web interface that allows the user to control and observe the simulation.
		Marketplace, merchants, consumer, and producer can be configured with the management UI.
		With this level of control, it is possible to test how merchants react to a changing market. E.g. how fast adapts a merchant its behavior if the number of consumers doubles.
		Various charts show merchant and consumer actions as well as merchants' short- and long-term performance.
	\item [Kafka Reverse Proxy]
	%todo market situation = snapshot of offers
		This services provides merchants with data about past market situations and sales.
		The data is filtered --- a merchant gets only information about his own sales --- and transformed into the CSV format.
		Additionally, the management UI gets continuous updates for its charts from the kafka reverse proxy.
		
\end{description}

Additionally, the services Kafka, Zookeeper, Flink, Postgres, and Redis are used to store and process data.
%This sentence is to general. Maybe explain each service in detail.
The platform's source code and documentation is available on \url{https://github.com/hpi-epic/pricewars}.
