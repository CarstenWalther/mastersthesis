%don't mention platform except in impl
\chapter{Merchant}
%is propose right word here?
This thesis proposes a merchant software agent that can compete against competition on an online marketplaces.
The merchant makes ordering and pricing decisions with the goal to maximize the profit.
The merchant uses demand learning to estimate future customer demand from historical market situations and sales data.
The estimated demand is used in a dynamic programming approach to generate ordering and pricing policies.
These policies define for every situation an appropriate action.
%what are those situations? what is an appropiate action?

% other word for step? phases? scenarios?
%todo be more concrete
To overcome the complexity of this problem, we developed the merchant over four steps.
The first step is a constrained scenario.
Each following step adds to the merchant's problem but makes the scenario more realistic.

\todo{explain each ''step'' short in some sentences each}
This is a test sentence as can be seen \cref{section:ordering_policy}.

All steps are explained in detail in the following sections.

\section{Ordering Policy}
\label{section:ordering_policy}
\subsection{Model}
%not continuous time, but period can be made arbitrary small
\subsection{Implementation}
\subsection{Evaluation}

\section{Joint Ordering and Pricing}
\subsection{Model}
\subsection{Implementation}
\subsection{Evaluation}

\section{Demand Learning}
\subsection{Model}
\subsection{Implementation}
\subsection{Evaluation}

\section{Competition}
\subsection{Model}
\subsection{Implementation}
\subsection{Evaluation}

%extra section for adaptive dynamic programming?