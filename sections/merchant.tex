%don't mention platform except in impl
\chapter{Merchant}
%is propose right word here?
This thesis proposes a merchant software agent that can compete against competition on an online marketplaces.
The merchant makes ordering and pricing decisions with the goal to maximize the profit.
The merchant uses demand learning to estimate future customer demand from historical market situations and sales data.
The estimated demand is used in a dynamic programming approach to generate ordering and pricing policies.
These policies define for every situation an appropriate action.
%what are those situations? what is an appropiate action?

% other word for step? phases? scenarios?
%todo be more concrete
To overcome the complexity of this problem, we developed the merchant over four steps.
The first step is a constrained scenario.
Each following step adds to the merchant's problem but makes the scenario more realistic.

\todo{explain each ''step'' short in some sentences each} \cref{section:ordering_policy}.

All steps are explained in detail in the following sections.
%explain structure of following sections

\section{Ordering Policy}
\label{section:ordering_policy}
In this scenario, the merchant cannot make pricing decisions.
Instead, all products are sold to a fixed price.
The merchant can focus on the ordering problem and make ordering decisions that promise the most expected long-term profit.
The merchant has a monopoly and does not have to care about any competitor.
Ordering decisions depend on the customer demand.
The demand is stochastic and its distribution is known to the merchant.
There are no backlogs. %sentence to short?
If a customer arrives when the merchant is out of stock, the merchant will miss the sale.

\subsection{Model Description}
%situation
%todo more intro/situation
The merchant wants to sell items on the marketplace.
% price & revenue
Items are offered at a fixed price $p$ and each sold item creates a revenue of $p$.
% time horizon
The time horizon $T$ is infinite.
% discrete time
Since in real-life applications prices cannot be adjusted arbitrarily often, we use a discrete time
model.
%discounting
We use discounting to increase the relevance of short-term profits.
A discount factor $\delta$, $0 < \delta \leq 1$, is applied to each time period.

% inventory + holding cost
%todo introduce time and periods before that
The merchant holds items in an inventory.
The random inventory level at the start of period t is denoted by $N_t$.
Storing items in the inventory causes holding costs of $l$ per item per time period. % l > 0

%ordering
The merchant can reorder items to increase the inventory level.
Order decisions, which can be made once each time period, will influence the merchant's profit.
The number of items ordered at time $t$ are denoted by $b_t$, with $b_t \geq 0$.
A order of size $b_t = 0$ means that no order is made.
%todo order delay
%set depends on time, yes? no?
The set of admissible orders quantities is denoted $B_t$.
% order cost
% explain fixed/variable order cost? explain somewhere else?
Each order causes a order cost, which consists of fixed order cost $c_{fix}$ and variable order cost $c_{var}$.
%order costs paid upfront
Order costs are defined by:

%why big C for function?
$$
C(b) := \begin{cases}
c_{fix} + c_{var} * b  & \quad \text{if } b > 0 \\
0  & \quad \text{if } b = 0
\end{cases}
$$

%sales
The probability to sell $i$ items is denoted by $P(i)$.
%check: myopic somewhere mentioned and explained
Note, that the probability is time-independent because of the consumer's myopic buying behavior.
Additionally, the probability in this scenario is independent from offer prices because the merchants sells in a monopoly with a fixed price.
The random number of sold items within the period $(t-1, t)$ is denoted by $X_t$.
%check: backorders mentioned in previous section
%can I express the line below with random variables?
The merchant cannot sell more items than the inventory holds, i.e. $N_t \geq X_{t+1}$.

%whats ordering strategy?
%how to write the policy?
Depending on a given ordering policy $(b_t)_t$, the random accumulated profit from time period $t$ is:

$$
G_t := \sum_{s=t+1}^{T} (\delta^{s-t} * (p * X_s - l * N_{s-1} - C(b_{s-1}(N_{s-1})))
$$

The objective is to find a ordering policy that maximizes the expected total profit $E(G_0 | N_0)$.

% end of model descripiton, maybe write what comes next
%Sales decrease the inventory level over time and orders increase it.

%todo lösungsansatz subsection
Ordering decisions depend on the demand distribution, the fixed and variable order cost, the holding cost rate, and the selling price.
We use dynamic programming to calculate optimal ordering policies.
\todo{cite paper why it's optimal}

%explain how dyn programming applied here
This approach loses some precision, because the merchant decides in discrete time instead of continuous time.
But the policies come arbitrarily close to the optimal continuous solution by shortening the decision interval.
% plus shipping time

% specify model parameter somewhere
%does selling price influence policy? yes, because stochastic demand?

\subsection{Implementation}
%evtl nur einmal implementation für kompletten merchant
\subsection{Evaluation}
%compare with merchant that buys always one more / one less

\section{Joint Ordering and Pricing}
\subsection{Model}
%model from above gets extended by pricing decision
%define set of pricing decisions
\subsection{Implementation}
\subsection{Evaluation}

\section{Demand Learning}
\subsection{Model}
\subsection{Implementation}
\subsection{Evaluation}

\section{Competition}
\subsection{Model}
\subsection{Implementation}
\subsection{Evaluation}

%extra section for adaptive dynamic programming? -> runs in 0.1 sec