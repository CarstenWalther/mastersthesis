%don't mention platform except in impl
\chapter{Merchant}
%is propose right word here?
This thesis proposes a merchant software agent that can compete against competition on an online marketplaces.
The merchant makes ordering and pricing decisions with the goal to maximize the profit.
The merchant uses demand learning to estimate future customer demand from historical market situations and sales data.
The estimated demand is used in a dynamic programming approach to generate ordering and pricing policies.
These policies define for every situation an appropriate action.
%what are those situations? what is an appropiate action?

% other word for step? phases? scenarios?
%todo be more concrete
To overcome the complexity of this problem, we developed the merchant over four steps.
The first step is a constrained scenario.
Each following step adds to the merchant's problem but makes the scenario more realistic.

\todo{explain each ''step'' short in some sentences each} \cref{section:ordering_policy}.

All steps are explained in detail in the following sections.
%maybe explain structure of following sections?

\section{Ordering Policy}
\label{section:ordering_policy}
In this scenario, the merchant cannot make pricing decisions.
Instead, all products are sold to a fixed price.
The merchant can focus on the ordering problem and make ordering decisions that promise the most expected long-term profit.
The merchant has a monopoly and does not have to care about any competitor.
Ordering decisions depend on the customer demand.
The demand is stochastic and its distribution is known to the merchant.
There are no backlogs. %sentence to short?
If a customer arrives when the merchant is out of stock, the merchant will miss the sale.

\subsection{Model}
The merchant's sale is an ongoing process on the marketplace, which can be controlled with ordering decisions.
Sales and ordering decisions influence the merchant's state, the inventory level.
Sales decrease the inventory level over time and orders increase it.
Ordering decisions depend on the demand distribution, the fixed and variable order cost, the holding cost rate, and the selling price.
We use dynamic programming to calculate optimal ordering policies.
\todo{cite paper why it's optimal}

%explain how dyn programming applied here
This approach loses some precision, because the merchant decides in discrete time instead of continuous time.
But the policies come arbitrarily close to the optimal continuous solution by shortening the decision interval.
% plus shipping time

% specify model parameter somewhere
%does selling price influence policy? yes, because stochastic demand?
\subsection{Implementation}
\subsection{Evaluation}
%compare with merchant that buys always one more / one less

\section{Joint Ordering and Pricing}
\subsection{Model}
\subsection{Implementation}
\subsection{Evaluation}

\section{Demand Learning}
\subsection{Model}
\subsection{Implementation}
\subsection{Evaluation}

\section{Competition}
\subsection{Model}
\subsection{Implementation}
\subsection{Evaluation}

%extra section for adaptive dynamic programming? -> runs in 0.1 sec