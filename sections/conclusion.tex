
\chapter{Conclusion \& Future Work}
\label{chapter:conclusion}

%%%%policy optimization
%restate problem, why important:
%what decisions?
Making suitable decisions on competitive online marketplaces is crucial for a merchant's success.
%solution
We presented different dynamic optimization models for four problem scenarios, which become increasingly realistic but also more complex.
The final problem scenario considers ordering and pricing decisions on an online marketplace under competition.
Pricing and ordering policies are created with a dynamic programming approach.
The uncertain demand is estimated based on historical market data using demand learning with linear regression.

%key findings / main points
We implemented a merchant that uses the proposed optimization approaches.
The merchant was tested and evaluated on the \pricewars platform.
Computation time efficiency of the dynamic programming approach was improved until the merchant could react to new market situations within one second.
This time can be further reduced with little impact on resulting policies.
One advantage of our adaptive dynamic programming approach is that suitable ranges of potential pricing and ordering decisions are found over time and do not have to be configured by the user.
Our merchant can cope with demand changes over time because of demand learning and the adaptable sets of pricing and ordering decisions. 

We compared performance of different merchant strategies in duopoly and oligopoly simulations.
As soon as a sufficient amount of sales observations is available, our merchant outperforms traditional rule-based merchants.
In duopoly settings, our merchant typically undercuts the competitor if it is not profitable to have a much higher or lower price.
Our merchant raises the price at some point, if both merchants undercut each other and the price decreases over time.
%significance of findings / application
The presented optimization approaches can be applied to real online marketplaces -- such as Amazon -- to potentially increase profits.
%DISCLAIMER: Trading is on your own risk. Please do not sue us if making losses with the presented approaches.

%%%%%platform extension
%problem
Formerly, it was not possible to simulate joint ordering and pricing problem scenarios on the \pricewars platform.
%solution
We extended the platform by orders of multiple items, fixed order costs, holding costs, and order delivery delay.
%impact of solution
These extensions allow testing and evaluating merchant strategies under more realistic conditions.
Moreover, our merchant implementation can be used as a baseline to compare new ordering and pricing strategies.

%%%%%future work
There are three lines of work arising from this thesis.
The dynamic pricing and inventory control problem can be extended to consider perishable products or multiple products with interdependent demand.
%make this sentence better
Both problems require changes to the dynamic programming approach and the problem with multiple products with interdependent demand requires changes to the demand learning component.
The price and order optimization could anticipate competitors' reactions.
This can result in different policies compared to optimizations that do not consider competitors' reactions.
Lastly, customer demand could depend on more dimensions than the price, like product quality or merchant rating.
This requires new explanatory variables for market situations and maybe a more complex demand learning model.

%market more dynamic -> be dynamic to compete -> better strat decides)

%so what?
%why should anyone care?

%finding: update interval (i.e. calculation speed) is key (needs eval)
%finding: policies optimal as long as sales prob prediction is good (-> reduces problem to find good predictions)