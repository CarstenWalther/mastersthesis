%!TEX root = ../thesis.tex

\chapter{Introduction}
%Why dynamic pricing?
Price updates on today's online markets increase in number and happen in shorter intervals.
Merchants want an advantage by adjusting their prices to competitors' prices.
The frequent updating of prices is called dynamic pricing and is used by merchants to increase their profit compared to traditional pricing strategies.
Human agents lack the fast response times and endurance to compete with software agents.
Additionally, these programs are able to analyze huge amounts of historical market data and can estimate customer buying behavior from it.
%Online marketplaces are ideal environments to apply dynamic pricing strategies because there is almost no cost for price changes.
%human irrational decisions

%Why inventory control?
Another important task of a merchant is keeping track of the inventory.
If the inventory level is to low, the merchant might miss potential sales due to a stock-out.
But storing to many items causes preventable extra costs.
This is the inventory control problem and it is about when to order how many new products.
It is crucial to accurately predict future demand for good order decisions.
%todo put following sentence somewhere after costs are introduced
%The merchant wants to find a good balance between few big orders, which minimizes fixed order costs, and many small orders, which minimizes holding costs.

%Joint pricing and inventory
Pricing and inventory decisions influence each other.
The optimal price depends not only on external factors like competitors' prices but also on the available inventory.
When inventory is low, it could be a good decision to increase the price in order to reduce the demand.
If the merchant reduces prices, the demand will increase, which must be considered in the order decision.
Because of this mutual influence, ordering and pricing should be decided jointly.

%platform + contributions
%todo contributions as compact list?
%joint ordering/pricing repetition
We developed a merchant that makes joint ordering and pricing decisions using dynamic programming with the goal to maximize its profit.
The merchant uses demand learning to estimate future demand based on data of previous sales.
This merchant runs on the \pricewars platform, a open framework to simulate dynamic pricing competition on online marketplaces.
% word 'settings' is ambiguous
The platform was created to test how merchant strategies perform in realistic settings.
This thesis extends the platform by the inventory control problem.
% formulate more positive 'force'
This forces merchants to control their inventory level and make order decisions.
This extension makes more realistic marketplace simulations possible.

%why not existing algorithms?
%	heuristic solutions -> performance can be improved
%	(no solution known with current ML techniques)
%why ML?
%	vielversprechende Ergebnisse in ähnlichen Gebieten
%why on Pricewars?
%	reproducable
%	comparable (to other merchants)

% motivation