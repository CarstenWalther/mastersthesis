%!TEX root = ../thesis.tex

\chapter{Introduction}
%Why dynamic pricing?
Price updates on today's online markets increase in number and happen in shorter intervals.
Merchants get an advantage by adjusting their prices to competitors' prices.
Frequently updating prices is called dynamic pricing and it is used by merchants to increase profit compared to traditional pricing strategies.
Online marketplaces are ideal environments for dynamic pricing strategies because price changes are cheap and automatable.
Software agents have faster response times and do not need to rest compared to human agents.
Additionally, these programs are able to analyze huge amounts of historical market data and can estimate customer buying behavior from it.

%Why inventory control?
Another important task of a merchant is keeping track of the inventory.
If the inventory level is too low, the merchant might miss potential sales due to a stock-out.
But storing too many items causes high holding costs.
This is the inventory control problem or ordering problem and it is about when to order how many items.
It is crucial to accurately predict future demand for good ordering decisions.

%Joint pricing and inventory
Managing prices and inventory is difficult because pricing and ordering decisions influence each other.
The optimal price depends not only on external factors like competitors' prices but also on the available inventory.
When inventory is low, it could be a good decision to increase the price in order to reduce the demand.
If the merchant reduces prices, the demand will increase, which must be considered in the order decision.
Because of this mutual influence, ordering and pricing should be decided jointly.

%contributions
Throughout this thesis, we make the following contributions:
\begin{itemize}
	\item We derived an optimization model for joint pricing and ordering problems (\cref{section:joint_ordering_pricing}). Optimal pricing and ordering policies are calculated with dynamic programming.
	\item We estimate customer demand based on historical market data using demand learning with linear regression (\cref{section:demand_learning}).
	\item We extended our model to cope with competing merchants (\cref{section:competition}).
	\item We improved computation time efficiency of our dynamic programming approach to allow quick merchant reactions to new market situations (\cref{section:faster_dyn_prog}).
	\item We implemented a merchant that utilizes our optimization models and runs on the  \pricewars platform (\cref{section:merchant_implementation}).
	\item We made numerical studies to show the applicability of our solution (\cref{section:order_example,section:convergence,section:joint_example,section:prediction_quality}).
	\item We benchmarked merchant strategies in duopoly (\cref{section:duopoly}) and oligopoly scenarios (\cref{section:oligopoly}). Our merchant consistently outperforms traditional rule-based merchants.
	\item Further, simulations showed that undercutting competitors is advantageous but only with sporadic price raises to increase the overall price level (\cref{section:duopoly,section:oligopoly}).
\end{itemize}

%price wars platform
The \pricewars platform is an open framework to simulate dynamic pricing competition on online marketplaces.
We used the platform to test and evaluate the performance of our merchant strategy.
An extension of the platform was necessary to be able to simulate joint pricing and ordering problem scenarios on it.
We implemented a more realistic ordering and inventory control concept on the \pricewars platform.
This was done by extending the platform by orders of multiple items, fixed order costs, inventory holding costs, and order delivery delay.
Tests on the platform are comparable and reproducible and since the platform is open-source software, researchers and practitioners can test their merchant strategies in the same or similar problem scenarios.

%paper structure

%related work
%price wars platform
%optimization models in 4 phases
%extension of platform
%conclusion future work

%why not existing algorithms?
%	heuristic solutions -> performance can be improved
%	(no solution known with current ML techniques)
%why ML?
%	vielversprechende Ergebnisse in ähnlichen Gebieten

% motivation
%mention applications for my solution