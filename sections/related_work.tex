
\chapter{Related Work}

Merchants on online marketplaces have to deal with two groups of decisions.
They decide when to order products and how many.
This is what the ordering problem is about.
Additionally, merchants decide offer prices for their products.
The problem of a suitable price choice is the pricing problem.

\subsubsection*{Ordering Problem}
%known demand
The ordering problem, also called inventory control problem, has been studied for a long time.
An overview about this problem for deterministic and stochastic demand is given by Scarf~\cite{scarf1963survey}.
Harris~\cite{harris1913many} proposed a solution for the ordering problem with constant deterministic demand and Wagner and Within~\cite{wagner1958dynamic} proposed a solution under deterministic but varying demand.

%unknown demand
If demand is uncertain, future demand must be estimated from past sales.
One approach is to investigate specific classes of parameterized demand distributions and propose methods to find parameters, so that the demand distribution fits the experienced sales best~\cite{azoury1985bayes}.
Other publications propose methods to find ordering policies without assumptions about the underlying demand distribution~\cite{DBLP:journals/mor/LeviRS07,huh2011adaptive,ban2017big}

Besbes and Muharremoglu~\cite{DBLP:journals/mansci/BesbesM13} study the effect of access to information about missed sales on the necessity to explore inventory decisions.

\subsubsection*{Pricing Problem}
%known demand
Each retailer has to face the problem of choosing a selling price.
Talluri and van Ryzin~\cite{talluri2004theory}, Phillips~\cite{phillips2005pricing}, and Yeoman and McMahon-Beattie~\cite{yeoman2010revenue} provide an extensive overview about solution approaches for this topic.

If it is allowed to frequently update prices, the problem is also called dynamic pricing problem.
Gallogo and van Ryzin~\cite{gallego1994optimal} proposed an optimal solution if demand is exponentially distributed.

%unknown demand
In real-world applications, demand is typically uncertain, but can be estimated based on past sales.
Den Boer~\cite{den2015dynamic} surveys literature about this topic.
Different approaches to deal with uncertain demand are Bayesian estimation~\cite{araman2011revenue}, maximum likelihood estimation~\cite{DBLP:journals/ior/BroderR12}, and least squares estimation~\cite{le2008data} among others.

\subsubsection*{Joint Ordering and Pricing Problem}
%known demand
Joint ordering and pricing is a combination of both problems and is a common problem in retail.
One challenge is that ordering and pricing decisions affect each other.
Elmaghraby and Keskinocak~\cite{elmaghraby2003dynamic} review literature about the joint ordering and pricing problem.
Solutions are proposed for different problem scenarios, if demand is known~\cite{thomas1970price,DBLP:journals/ior/FedergruenH99,chen2003coordinating,simchi2014integration}.

%unknown demand
If demand is uncertain, it must be estimated from past market observations.
Bitran and Wadhwa~\cite{bitran1996methodology} and Bisi and Dada~\cite{bisi2007dynamic} propose Bayesian based approaches for the ordering and pricing problem with uncertain demand.
Adida and Perakis~\cite{DBLP:journals/mp/AdidaP06,DBLP:journals/anor/AdidaP10} study this problem in a multi-product scenario without backordering.

\subsubsection*{Pricing and Ordering under Competition}
%only pricing
Chen and Chen~\cite{chen2015recent} provide an overview about the dynamic pricing problem under competition for single-product and multi-product scenarios.
Schlosser and Boissier~\cite{Schlosser_2017} proposed a method to find optimal pricing policies if competitor strategies are known.
Dynamic pricing problems under competition with a finite time horizon have been studied~\cite{DBLP:journals/mansci/Martinez-de-AlbenizT11,Schlosser_2018}.
%joint pricing & inventory 
Adida and Perakis~\cite{DBLP:journals/ior/AdidaP10} consider joint pricing and inventory control in a duopoly.
%todo hier könnte noch mehr

\subsubsection*{Simulation Platforms for Pricing Competition}
For testing and evaluating merchant strategies, we use the \pricewars platform~\cite{edoc2017pricewars}, a framework to simulate dynamic pricing competition on online marketplaces.
We chose this platform over other solutions (e.g.,~\cite{morris2001simulation,DBLP:journals/ecr/DiMiccoMG03,DBLP:journals/icae/PintoVSPSM14}) because its continuous time model makes the platform similar to real online marketplaces like Amazon.
Additionally, users are unrestricted in their choice and implementation of merchant strategies.
%(controllable customer behavior)

%Work that proposes a different method to solve the same problem.
%Work that uses the same proposed method to solve a different problem.
%A method that is similar to your method that solves a relatively similar problem.
%A discussion of a set of related problems that covers your problem domain.